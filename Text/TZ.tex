%!TEX TS-program = xelatex

% Шаблон документа LaTeX создан в 2018 году
% Алексеем Подчезерцевым
% В качестве исходных использованы шаблоны
% 	Данилом Фёдоровых (danil@fedorovykh.ru) 
%		https://www.writelatex.com/coursera/latex/5.2.2
%	LaTeX-шаблон для русской кандидатской диссертации и её автореферата.
%		https://github.com/AndreyAkinshin/Russian-Phd-LaTeX-Dissertation-Template

\documentclass[a4paper,14pt]{article}

\input{data/preambular.tex}
\begin{document} % конец преамбулы, начало документа
\begin{titlepage}
	\begin{center}
		МОСКОВСКИЙ ИНСТИТУТ ЭЛЕКТРОНИКИ И МАТЕМАТИКИ\\
		ИМ. А.Н.ТИХОНОВА НАЦИОНАЛЬНОГО ИССЛЕДОВАТЕЛЬСКОГО\\
		УНИВЕРСИТЕТА ВЫСШАЯ ШКОЛА ЭКОНОМИКИ
	\end{center}
	
	
	\vspace{8ex}
	
	\begin{flushright}
		Утверждаю:\\
		\rule{2.5cm}{0.4pt} Полесский С.Н.\\
		"\rule{1.2cm}{0.4pt}"\rule{2.9cm}{0.4pt} \the\year \, г.
		
		
	\end{flushright}	
	
	\vspace{5ex}
	
	\begin{center}
		ТЕХНИЧЕСКОЕ ЗАДАНИЕ\\
		на разработку комплексного электронного макета\\
		радиоэлектронного средства <<Лампа с диммером>>\\
		по курсу <<Автоматизация проектных работ>>

	\end{center}	

	\vspace{2ex}
	\vfill
	
	\vspace{2ex}
	
	

	\vspace{5ex}
	\begin{center}
		Москва \the\year \, г.
	\end{center}
	
\end{titlepage}
\addtocounter{page}{1}
\tableofcontents
\pagebreak

\section*{ВВЕДЕНИЕ}

Данный документ является техническим заданием на разработку комплексного электронного макета радиоэлектронного средства <<THISname>>.

Документ составляется в соответствии с ГОСТ 19.201-78: <<Техническое задание. Требования к содержанию и оформлению>>.
\newpage 

\section{ОСНОВАНИЕ ДЛЯ РАЗРАБОТКИ}

<<THISname>> разрабатывается на основании учебного плана образовательной программы <<Информатика и вычислительная техника>> Московского Института Электроники и Математики им. А.Н. Тихонова по дисциплине <<Автоматизация проектных работ>>. 
\newpage 

\section{НАЗНАЧЕНИЕ РАЗРАБОТКИ}
\subsection{НАЗНАЧЕНИЕ СИСТЕМЫ}

Диммер является TODO устройством, позволяющим регулировать яркость светодиодов.	

\newpage

\subsection{ЦЕЛЬ РАЗРАБОТКИ}

Целью разработки является создание TODO устройства, позволяющего регулировать яркость светодиодов.	

\newpage

\section{ТЕХНИЧЕСКИЕ ХАРАКТЕРИСТИКИ}

Напряжение питания -- ~220В

Материал корпуса -- пластик

\newpage


\section{СХЕМА}

\begin{figure}[H]
	\centering
	\includegraphics[width=1\linewidth]{image/sh.jpg}
	\caption{Схема диммера}
	\label{fig:sh}
\end{figure}

\newpage

\section{ПРИНЦИП РАБОТЫ}

%http://samlib.ru/m/makeew_l_a/1615.shtml

\newpage
\section{ТРЕБОВАНИЯ}

Основные требования при разработке устройства:

\begin{itemize}
	\item Дешевизна;
	\item Надежность;
	\item Малогабаритность;
	\item Удобство использования.
\end{itemize}


%\newpage
%\renewcommand{\refname}{{\normalsize СПИСОК %ИСПОЛЬЗОВАННЫХ ИСТОЧНИКОВ}} 
%\centering 
%\begin{thebibliography}{9} 
%	\addcontentsline{toc}{section}{\refname} 
%	\bibitem{sql} 
%\end{thebibliography}

\end{document} % конец документа